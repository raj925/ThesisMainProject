% Options for packages loaded elsewhere
\PassOptionsToPackage{unicode}{hyperref}
\PassOptionsToPackage{hyphens}{url}
%
\documentclass[
]{article}
\usepackage{amsmath,amssymb}
\usepackage{iftex}
\ifPDFTeX
  \usepackage[T1]{fontenc}
  \usepackage[utf8]{inputenc}
  \usepackage{textcomp} % provide euro and other symbols
\else % if luatex or xetex
  \usepackage{unicode-math} % this also loads fontspec
  \defaultfontfeatures{Scale=MatchLowercase}
  \defaultfontfeatures[\rmfamily]{Ligatures=TeX,Scale=1}
\fi
\usepackage{lmodern}
\ifPDFTeX\else
  % xetex/luatex font selection
\fi
% Use upquote if available, for straight quotes in verbatim environments
\IfFileExists{upquote.sty}{\usepackage{upquote}}{}
\IfFileExists{microtype.sty}{% use microtype if available
  \usepackage[]{microtype}
  \UseMicrotypeSet[protrusion]{basicmath} % disable protrusion for tt fonts
}{}
\makeatletter
\@ifundefined{KOMAClassName}{% if non-KOMA class
  \IfFileExists{parskip.sty}{%
    \usepackage{parskip}
  }{% else
    \setlength{\parindent}{0pt}
    \setlength{\parskip}{6pt plus 2pt minus 1pt}}
}{% if KOMA class
  \KOMAoptions{parskip=half}}
\makeatother
\usepackage{xcolor}
\usepackage[margin=1in]{geometry}
\usepackage{color}
\usepackage{fancyvrb}
\newcommand{\VerbBar}{|}
\newcommand{\VERB}{\Verb[commandchars=\\\{\}]}
\DefineVerbatimEnvironment{Highlighting}{Verbatim}{commandchars=\\\{\}}
% Add ',fontsize=\small' for more characters per line
\usepackage{framed}
\definecolor{shadecolor}{RGB}{248,248,248}
\newenvironment{Shaded}{\begin{snugshade}}{\end{snugshade}}
\newcommand{\AlertTok}[1]{\textcolor[rgb]{0.94,0.16,0.16}{#1}}
\newcommand{\AnnotationTok}[1]{\textcolor[rgb]{0.56,0.35,0.01}{\textbf{\textit{#1}}}}
\newcommand{\AttributeTok}[1]{\textcolor[rgb]{0.13,0.29,0.53}{#1}}
\newcommand{\BaseNTok}[1]{\textcolor[rgb]{0.00,0.00,0.81}{#1}}
\newcommand{\BuiltInTok}[1]{#1}
\newcommand{\CharTok}[1]{\textcolor[rgb]{0.31,0.60,0.02}{#1}}
\newcommand{\CommentTok}[1]{\textcolor[rgb]{0.56,0.35,0.01}{\textit{#1}}}
\newcommand{\CommentVarTok}[1]{\textcolor[rgb]{0.56,0.35,0.01}{\textbf{\textit{#1}}}}
\newcommand{\ConstantTok}[1]{\textcolor[rgb]{0.56,0.35,0.01}{#1}}
\newcommand{\ControlFlowTok}[1]{\textcolor[rgb]{0.13,0.29,0.53}{\textbf{#1}}}
\newcommand{\DataTypeTok}[1]{\textcolor[rgb]{0.13,0.29,0.53}{#1}}
\newcommand{\DecValTok}[1]{\textcolor[rgb]{0.00,0.00,0.81}{#1}}
\newcommand{\DocumentationTok}[1]{\textcolor[rgb]{0.56,0.35,0.01}{\textbf{\textit{#1}}}}
\newcommand{\ErrorTok}[1]{\textcolor[rgb]{0.64,0.00,0.00}{\textbf{#1}}}
\newcommand{\ExtensionTok}[1]{#1}
\newcommand{\FloatTok}[1]{\textcolor[rgb]{0.00,0.00,0.81}{#1}}
\newcommand{\FunctionTok}[1]{\textcolor[rgb]{0.13,0.29,0.53}{\textbf{#1}}}
\newcommand{\ImportTok}[1]{#1}
\newcommand{\InformationTok}[1]{\textcolor[rgb]{0.56,0.35,0.01}{\textbf{\textit{#1}}}}
\newcommand{\KeywordTok}[1]{\textcolor[rgb]{0.13,0.29,0.53}{\textbf{#1}}}
\newcommand{\NormalTok}[1]{#1}
\newcommand{\OperatorTok}[1]{\textcolor[rgb]{0.81,0.36,0.00}{\textbf{#1}}}
\newcommand{\OtherTok}[1]{\textcolor[rgb]{0.56,0.35,0.01}{#1}}
\newcommand{\PreprocessorTok}[1]{\textcolor[rgb]{0.56,0.35,0.01}{\textit{#1}}}
\newcommand{\RegionMarkerTok}[1]{#1}
\newcommand{\SpecialCharTok}[1]{\textcolor[rgb]{0.81,0.36,0.00}{\textbf{#1}}}
\newcommand{\SpecialStringTok}[1]{\textcolor[rgb]{0.31,0.60,0.02}{#1}}
\newcommand{\StringTok}[1]{\textcolor[rgb]{0.31,0.60,0.02}{#1}}
\newcommand{\VariableTok}[1]{\textcolor[rgb]{0.00,0.00,0.00}{#1}}
\newcommand{\VerbatimStringTok}[1]{\textcolor[rgb]{0.31,0.60,0.02}{#1}}
\newcommand{\WarningTok}[1]{\textcolor[rgb]{0.56,0.35,0.01}{\textbf{\textit{#1}}}}
\usepackage{graphicx}
\makeatletter
\def\maxwidth{\ifdim\Gin@nat@width>\linewidth\linewidth\else\Gin@nat@width\fi}
\def\maxheight{\ifdim\Gin@nat@height>\textheight\textheight\else\Gin@nat@height\fi}
\makeatother
% Scale images if necessary, so that they will not overflow the page
% margins by default, and it is still possible to overwrite the defaults
% using explicit options in \includegraphics[width, height, ...]{}
\setkeys{Gin}{width=\maxwidth,height=\maxheight,keepaspectratio}
% Set default figure placement to htbp
\makeatletter
\def\fps@figure{htbp}
\makeatother
\setlength{\emergencystretch}{3em} % prevent overfull lines
\providecommand{\tightlist}{%
  \setlength{\itemsep}{0pt}\setlength{\parskip}{0pt}}
\setcounter{secnumdepth}{-\maxdimen} % remove section numbering
\usepackage{booktabs}
\usepackage{longtable}
\usepackage{array}
\usepackage{multirow}
\usepackage{wrapfig}
\usepackage{float}
\usepackage{colortbl}
\usepackage{pdflscape}
\usepackage{tabu}
\usepackage{threeparttable}
\usepackage{threeparttablex}
\usepackage[normalem]{ulem}
\usepackage{makecell}
\usepackage{xcolor}
\ifLuaTeX
  \usepackage{selnolig}  % disable illegal ligatures
\fi
\usepackage{bookmark}
\IfFileExists{xurl.sty}{\usepackage{xurl}}{} % add URL line breaks if available
\urlstyle{same}
\hypersetup{
  hidelinks,
  pdfcreator={LaTeX via pandoc}}

\author{}
\date{\vspace{-2.5em}}

\begin{document}

\startappendices

\section{Vignette Marking Scheme (Studies 2 and
3)}\label{vignette-marking-scheme-studies-2-and-3}

\begin{longtable}[t]{llll}
\toprule
Condition & Abbreviation & Presenting Complaint & Accepted Answers\\
\midrule
Temporal Arteritis & TA & Patient is a 68 year old male presented with fever and arthralgia. & Any inflammatory arthritis is accepted\\
Ulcerative Colitis & UC & Patient is a 60 year old male presented with 2 day history of bloody diarrhoea. & Infectious colitis, ischemic colitis and diverticulitis are also accepted answers.\\
Miliary Tuberculosis & MTB & Patient is a 62 year old male admitted for fevers and generalised weakness. & Any TB or lymphoma type is accepted\\
Aortic Dissection & AD & Patient is a 58 year old female presented with shortness of breath. & Pulmonary embolism and coarctation of the aorta are also accepted answers. Aortic stenosis\\
Guillain-Barré Syndrome & GBS & Patient is a 67 year old male presented with weakness of the legs for 24 hours. & Cauda Equina Syndrome is also accepted\\
\addlinespace
Thrombotic Thrombocytopenic Purpura & TTP & Patient is a 20 year old male was admitted from an outside hospital with complaints of a headache and slurred speech. & ITP or Meningitis are also accepted.\\
\bottomrule
\end{longtable}

\emph{Table S1: Marking scheme used to denote differentials that are
considered as correct for each of the six patient cases/vignettes. The
same marking scheme is applied for online and think-aloud vignette
studies. The presenting complaint is shown to participants at the start
of the case, before they start seeking information.}

\newpage

\section{Vignette Information
Requests}\label{vignette-information-requests}

\begin{longtable}[t]{lll}
\toprule
Patient History & Physical Examinations & Testing\\
\midrule
History of Presenting Complaint & Take Pulse & Urine Dipstick\\
Past Medical History & Measure Blood Pressure & ECG\\
Medications & Assess Respiratory Rate & Abdominal CT Scan\\
Allergies & Auscultate Lungs & Venous Blood Gas\\
Family History & Auscultate the Heart & CRP and ESR\\
\addlinespace
Social History & Assess Eyes & Clotting Test\\
 & Measure Temperature & FBC\\
 & Abdomen Examination & Other Biochemistry tests\\
 & Rectal Examination & UREA and Electrolytes\\
 & Neck/Throat Examination & Chest X-Ray\\
\addlinespace
 & Assess Head & \\
 & Neurological Exam Record & \\
 & Assess Extremities & \\
\bottomrule
\end{longtable}

\emph{Table S2: Full list of possible information requests that
participants can make. This set of information is the same for all
cases. The same vignettes and corresponding information are used for the
online and think-aloud vignette studies.}

\section{Calibration of Confidence to Alternative Accuracy
Measures}\label{calibration-of-confidence-to-alternative-accuracy-measures}

\subsection{Differential Accuracy}\label{differential-accuracy}

When comparing Differential Accuracy (if a correct differential is
provided in the participant's list) to Confidence, we find, across
stages, participants' Confidence was not aligned to their Accuracy.
Instead, we find evidence of underconfidence at all stages. There was
evidence of a significant difference between the two at the Patient
History (t(84) = 8.24, MDiff = 0.24, p = 0), Physical Examination stage
(t(84) = -9.09, MDiff = -0.25, p = 0), and Testing stage (t(84) = -7.74,
MDiff = -0.22, p = 0).

In order to examine the observed underconfidence in more detail, we
compare confidence and Differential Accuracy by case (the mean values of
which can be found in Table 1 of the main thesis). We conducted paired
t-tests for each condition's cases by comparing Differential Accuracy
and confidence values (at the final Testing stage) to observe if they
significantly differ from each other. A p value of less than .05 is
interpreted as evidence for overconfidence or underconfidence (depending
on the direction of the effect). We observed underconfidence for the GBS
case (t(84) = -7.43, MDiff = -0.39, p = \textless{} .001), the TA case
(t(84) = -5.07, MDiff = -0.25, p = \textless{} .001), the TTP case
(t(84) = -3.23, MDiff = -0.2, p = \textless{} .001) and the UC case
(t(82) = -14.83, MDiff = -0.38, p = \textless{} .001). The remaining
cases did not yield a significant effect.

\subsection{Highest Likelihood
Accuracy}\label{highest-likelihood-accuracy}

When comparing Highest Likelihood Accuracy (likelihood assigned to the
highest likelihood differential if it is correct) to Confidence, we
find, across stages, participants' Confidence was not aligned to their
Accuracy. Instead, we find evidence of overconfidence at all stages.
There was evidence of a significant difference between the two at the
Patient History (t(84) = -2.49, MDiff = -0.05, p = 0.01), Physical
Examination stages (t(84) = 4.45, MDiff = 0.09, p = 0), and Testing
stage (t(84) = 6.84, MDiff = 0.16, p = 0).

In order to examine the observed overconfidence in more detail, we
compare confidence and Highest Likelihood Accuracy by case (the mean
values of which can be found in Table 1 of the main thesis). We
conducted paired t-tests for each condition's cases by comparing Highest
Likelihood Accuracy and confidence values (at the final Testing stage)
to observe if they significantly differ from each other. A p value of
less than .05 is interpreted as evidence for overconfidence or
underconfidence (depending on the direction of the effect). We observed
overconfidence for the AD case (t(84) = 8.92, MDiff = 0.37, p =
\textless{} .001), the MTB case (t(83) = 7.66, MDiff = 0.35, p =
\textless{} .001) and the TTP case (t(84) = 4.09, MDiff = 0.21, p =
\textless{} .001). The remaining cases did not yield a significant
effect.

\section{Debrief Questionnaire from Think-Aloud Study (Study
3)}\label{debrief-questionnaire-from-think-aloud-study-study-3}

Each question has a corresponding follow-up question below in case they
are not answered by responses to the main questions.

\begin{itemize}
\tightlist
\item
  \begin{enumerate}
  \def\labelenumi{\arabic{enumi}.}
  \tightlist
  \item
    What's your general approach to making diagnoses? \emph{Follow-Up:}
    Do you have those cognitive aids or frameworks you use?
  \end{enumerate}
\item
  \begin{enumerate}
  \def\labelenumi{\arabic{enumi}.}
  \setcounter{enumi}{1}
  \tightlist
  \item
    Do you tend to keep a broad set of differentials in mind?
    \emph{Follow-Up:} Are there particular situations where having a
    narrower set would be more useful?
  \end{enumerate}
\item
  \begin{enumerate}
  \def\labelenumi{\arabic{enumi}.}
  \setcounter{enumi}{2}
  \tightlist
  \item
    How do you decide what information or tests to get on a patient?
    \emph{Follow-Up:} Would you say you tend to seek information to
    confirm or to rule out differentials that you have in mind?
  \end{enumerate}
\item
  \begin{enumerate}
  \def\labelenumi{\arabic{enumi}.}
  \setcounter{enumi}{3}
  \tightlist
  \item
    How similar was your diagnostic reasoning on this task versus how
    you would approach diagnosis in real life? \emph{Follow-Up:} Was
    there anything that prevented you from approaching the task as you
    would in real life?
  \end{enumerate}
\end{itemize}

\section{Diagnostic Appropriateness Marking Scheme (Study
3)}\label{diagnostic-appropriateness-marking-scheme-study-3}

\section{R Environment and Packages}\label{r-environment-and-packages}

\begin{Shaded}
\begin{Highlighting}[]
\CommentTok{\# print("R version:")}
\CommentTok{\# version$version.string}
\CommentTok{\# }
\CommentTok{\# print("Rstudio version:")}
\CommentTok{\# rstudioversion \textless{}{-} rstudioapi::versionInfo()}
\CommentTok{\# rstudioversion$version}
\CommentTok{\# }
\CommentTok{\# print("Citations for packages used:")}
\CommentTok{\# get\_pkgs\_info(pkgs = required\_packages, out.dir = getwd())}
\CommentTok{\# pkgs \textless{}{-} scan\_packages()}
\CommentTok{\# get\_citations(pkgs$pkg, out.dir = getwd(), include.RStudio = TRUE)}
\CommentTok{\# cite\_packages(pkgs = required\_packages, output = "table", out.format = "Rmd", out.dir = getwd())}
\CommentTok{\# }
\CommentTok{\# required\_packages \%\textgreater{}\%}
\CommentTok{\#   map(citation) \%\textgreater{}\%}
\CommentTok{\#   print(style = "text")}
\end{Highlighting}
\end{Shaded}


\end{document}
